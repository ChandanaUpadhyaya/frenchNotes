\section{Verbs and Conjugations}

There are three regular types of verbs in French:
\begin{enumerate}
\item{\textbf{Regular -ER Verbs:} e.g. parler, aimer, jouer etc.}
\item{\textbf{Regular -IR Verbs:} e.g. finir, \'etablir etc.}
\item{\textbf{Regular -RE Verbs:} e.g. attendre, vendre}
\end{enumerate}
Other than the above, there are irregular verbs such as \^Etre, Conna\^itre, Avoir.
Their conjugation methods are unique to the verbs themselves.

\subsection{Conjugation Pattern for Regular ER Verbs}

\subsubsection{Parler (to speak)}
\begin{tabular}{| l | r | l |}
\hline
Pronoun 	& 	Ending 	& 	Conjugation\\ \hline
Je		    &	-e	    & 	parle	\\ 	\hline
Tu		    &	-es	    &	parles	\\	\hline
Il/Elle		&	-e	    &	parle	\\	\hline
Nous		&	-ons	&	parlons	\\	\hline
Vous		&	-ez	    &	parlez	\\	\hline
Ils/Elles	&	-ent	&	parlent	\\	\hline
\end{tabular}

\subsubsection{List of Regular -ER Verbs}

\begin{longtable}{| l | l | l |}
\hline
Verb 		& Meaning 		& Pronunciation	\\
\hline
\endhead
parler		& to speak		& paarl			\\ 	\hline
jouer		& to play		& zooay			\\	\hline
aimer		& to like		& aimay			\\	\hline
\'ecouter	& to listen		& aikootey		\\	\hline
associer	& to associate	& associay		\\	\hline
regarder	& to look		& ghugarday		\\	\hline
tourner		& to turn		& toohnay		\\	\hline
fermer		& to close		& fahmay		\\	\hline
habiter		& to live		& aabitey		\\	\hline
adorer 		& to love		& adohay		\\	\hline
chercher	& to search		& shershay		\\	\hline
travailler	& to work		& travaiyay		\\	\hline
rester      & to stay       & rastay        \\  \hline
pratiquer   & to practice   & prahtikay     \\  \hline
inviter     & to invite     & aanvitay      \\  \hline
refuser     & to refuse     & ruhfyusay     \\  \hline
accepter    & to accept     & ekseptay      \\  \hline
visiter     & to visit      & veezeetay     \\  \hline
animer      & to host, lead & aneemay       \\  \hline
poser       & to put        & posay         \\  \hline
porter      & to wear, carry& portay        \\  \hline
terminer    & to end, terminate & tarminay  \\  \hline
utiliser    & to use        & yootileezayy  \\  \hline
\end{longtable}

\subsection{Conjugation pattern for -GER Verbs}
These are also called spelling change verbs. The conjugation pattern is
similar to regular -er verbs, but the only difference is the conjugation
with \emph{Nous}.

\subsubsection{Manger (to eat)}
\begin{tabular}{| l | l |}
\hline
Pronoun 	& 	Conjugation	\\
Je		&	mange		\\
Tu		&	manges		\\
Il/Elle		&	mange		\\
Nous		&	mangeons	\\
Vous		&	mangez		\\
Ils/Elles	&	mangent		\\
\hline
\end{tabular}

\subsubsection{List of -GER verbs}
\begin{longtable}{| l | l | l |}
\hline
Verb 		& Meaning 		& Pronunciation	\\
\hline
\endhead
voyager 	& to travel		& voyajay	\\  \hline
manger		& to eat		& manjay	\\  \hline
arranger	& to arrange	& amanjay	\\  \hline
nager       & to swim       & najay		\\  \hline
\end{longtable}


\subsection{Conjugation pattern for -ELER Verbs}
These are also called Stem-changing verbs. The ending pattern is similar
to regular -er verbs, but the ``ll'' in spelling is different in conjugations.

\subsubsection{Appeler (to call)}
\begin{tabular}{| l | l |}
\hline
Pronoun 	& 	Conjugation	\\
J'		&	appelle		\\
Tu		&	appelles	\\
Il/Elle		&	appelle		\\
Nous		&	appelons	\\
Vous		&	appelez		\\
Ils/Elles	&	appellent	\\
\hline
\end{tabular}

\subsubsection{List of -ELER verbs}
\begin{longtable}{| l | l | l |}
\hline
Verb 		& Meaning 		& Pronunciation	\\
\hline
\endhead
appeler 	& to call		& aapaylay	\\
\'epeler 	& to spell		& apaylay	\\
\hline
\end{longtable}

\subsection{Conjugation pattern for Regular -RE Verbs}

\subsubsection{R\'epondre (to answer)}
\begin{tabular}{| l | r | l |}
\hline
Pronoun 	& 	Ending 	& 	Conjugation	\\ 	\hline
Je		    &	-s	    & 	r\'eponds   \\ 	\hline
Tu		    &	-s	    &	r\'eponds   \\	\hline
Il/Elle		&	-	    &	r\'epond    \\	\hline
Nous		&	-ons	&	r\'epondons \\	\hline
Vous		&	-ez	    &	r\'epondez  \\	\hline
Ils/Elles	&	-ent	&	r\'epondent \\	\hline
\end{tabular}

\subsubsection{List of Regular -RE verbs}
\begin{longtable}{| l | l | l |}
\hline
Verb 		& Meaning 		& Pronunciation	\\
\hline
\endhead
perdre      & to lose       & purdurh       \\ 	\hline
vendre      & to sell       & voondurh      \\	\hline
r\'epondre  & to answer     & rapondruh     \\	\hline
\'entendre  & to hear       & auntaundruh   \\	\hline
descendre   & to descend    & deesaundruh   \\	\hline
d\'efendre   & to defend     & deefaundruh   \\	\hline
\end{longtable}

\subsection{Conjugation pattern for Regular -IR Verbs}

\subsubsection{Finir (to finish)}
\begin{tabular}{| l | r | l |}
\hline
Pronoun 	& 	Ending 	& 	Conjugation	\\ 	\hline
Je		    &	-is	    & 	finis       \\ 	\hline
Tu		    &	-is	    &	finis       \\	\hline
Il/Elle		&	-it	    &	finit       \\	\hline
Nous		&	-issons &	finissons   \\	\hline
Vous		&	-issez  &	finissez    \\	\hline
Ils/Elles	&	-issent	&	finissent   \\	\hline
\end{tabular}

\subsubsection{List of Regular -IR verbs}
\begin{longtable}{| l | l | l |}
\hline
Verb 		& Meaning 		& Pronunciation	\\
\hline
\endhead
finir       & to finish       & finir       \\	\hline
abolir      & to abolish      & abolir      \\	\hline
agir        & to act          & agir        \\	\hline
\'etablir   & to establish    & itablir     \\	\hline
\end{longtable}


\subsection{Other Irregular Verbs}

\subsubsection{S'appeler (to call oneself)}
\begin{tabular}{| l | l |}
\hline
Pronoun 	& 	Conjugation	\\
Je		&	m'appelle	\\
Tu		&	t'appelles	\\
Il/Elle		&	s'appelle	\\
Nous		&	nous appelons	\\
Vous		&	vous appelez	\\
Ils/Elles	&	s'appellent	\\
\hline
\end{tabular}

\subsubsection{\^Etre (to be)}
\begin{tabular}{| l | l |}
\hline
Pronoun 	& 	\^Etre	\\
Je		&	suis	\\
Tu		&	es	\\
Il/Elle		&	est	\\
Nous		&	sommes	\\
Vous		&	\^etes	\\
Ils/Elles	&	sont	\\
\hline
\end{tabular}

\subsubsection{Avoir (to have)}
\begin{tabular}{| l | l |}
\hline
Pronoun 	& 	Avoir	\\
Je		&	ai	\\
Tu		&	as	\\
Il/Elle		&	a	\\
Nous		&	avons	\\
Vous		&	avez	\\
Ils/Elles	&	ont	\\
\hline
\end{tabular}

\subsubsection{Connaitre (to know)}
\begin{tabular}{| l | l |}
\hline
Pronoun 	& 	Connaitre	\\
Je		    &	connais		\\
Tu		    &	connais		\\
Il/Elle     &	conna\^it	\\
Nous		&	connaissons	\\
Vous		&	connaissez	\\
Ils/Elles	&	connaissent	\\
\hline
\end{tabular}

\subsubsection{Comprendre (to understand)}
\begin{tabular}{| l | l |}
\hline
Pronoun 	& 	Conjugation	\\
Je		&	comprends	\\
Tu		&	comprends	\\
Il/Elle		&	comprend	\\
Nous		&	comprenons	\\
Vous		&	comprenez	\\
Ils/Elles	&	comprennent	\\
\hline
\end{tabular}

\subsubsection{Lire (to read)}
\begin{tabular}{| l | l |}
\hline
Pronoun 	& 	Conjugation	\\
Je		&	lis		\\
Tu		&	lis		\\
Il/Elle		&	lit		\\
Nous		&	lisons		\\
Vous		&	lisez		\\
Ils/Elles	&	lisent		\\
\hline
\end{tabular}

\subsubsection{\'Ecrire (to write)}
\begin{tabular}{| l | l |}
\hline
Pronoun 	& 	Conjugation	\\
Je		&	\'ecris		\\
Tu		&	\'ecris		\\
Il/Elle		&	\'ecrit		\\
Nous		&	\'ecrivons	\\
Vous		&	\'ecrivez	\\
Ils/Elles	&	\'ecrivent	\\
\hline
\end{tabular}


\subsubsection{Voir (to see)}
\begin{tabular}{| l | l |}
\hline
Pronoun 	& 	Conjugation	\\
Je		    &	voi\ul{s}	\\
Tu		    &	voi\ul{s}	\\
Il/Elle		&	voi\ul{t}	\\
Nous		&	voyons		\\
Vous		&	voyez		\\
Ils/Elles	&	voi\ul{ent}	\\
\hline
\end{tabular}

\subsubsection{Aller (to go)}
\begin{tabular}{| l | l |}
\hline
Pronoun 	& 	Conjugation	\\
Je		    &	vais		\\
Tu		    &	vas		\\
Il/Elle		&	va		\\
Nous		&	allons		\\
Vous		&	allez		\\
Ils/Elles	&	vont		\\
\hline
\end{tabular}

\vspace{0.2in}
\noindent \textbf{Aller is used in several idiomatic expressions :}\\\\
Je vais \`{a} pied.\\
I am going on foot.\\\\
\c{C}a va sans dire.\\
That goes without saying.\\\\
On y va ?\\
Shall we go?

\subsubsection{Venir (to come)}
\begin{tabular}{| l | l |}
\hline
Pronoun 	& 	Conjugation	\\
Je		    &	viens		\\
Tu		    &	viens		\\
Il/Elle		&	vient		\\
Nous		&	venons		\\
Vous		&	venez		\\
Ils/Elles	&	viennent		\\
\hline
\end{tabular}

\subsubsection{Faire (to do / to make)}
\begin{tabular}{| l | l |}
\hline
Pronoun 	& 	Conjugation	\\
Je		    &	fais		\\
Tu		    &	fais		\\
Il/Elle		&	fait		\\
Nous		&	faisons		\\
Vous		&	faites		\\
Ils/Elles	&	font		\\
\hline
\end{tabular}

\subsubsection{Vouloir (to want)}
\begin{tabular}{| l | l |}
\hline
Pronoun 	& 	Conjugation	\\
Je		    & veux        \\
Tu		    & veux        \\
Il/Elle		& veut        \\
Nous		& voulons     \\
Vous		& voulez      \\
Ils/Elles	& veulent     \\
\hline
\end{tabular}


\noindent \\Examples:\\
Je veux manger un g\^ateau.

\subsubsection{Pouvoir (to be able to, can)}
\begin{tabular}{| l | l |}
\hline
Pronoun 	& Conjugation	\\
Je		    & peux     \\
Tu		    & peux     \\
Il/Elle		& peut     \\
Nous		& pouvons     \\
Vous		& pouvez     \\
Ils/Elles	& peuvent     \\
\hline
\end{tabular}

\noindent \\Examples:\\
Je peux manger deux g\^ateaux.

\subsubsection{Savoir (to know)}
\begin{tabular}{| l | l |}
\hline
Pronoun 	& Conjugation	\\
Je		    &  sais \\
Tu		    &  sais \\
Il/Elle		&  sait \\
Nous		&  savons \\
Vous		&  savez  \\
Ils/Elles	&  savent \\
\hline
\end{tabular}

\noindent \\Examples:\\
Je sais, je suis grosse.

\subsubsection{Devoir(to have to, must, to owe)}
\begin{tabular}{| l | l |}
\hline
Pronoun 	& Conjugation	\\
Je		    &  dois \\
Tu		    &  dois \\
Il/Elle		&  doit \\
Nous		&  devons   \\
Vous		&  devez \\
Ils/Elles	&  doivent \\
\hline
\end{tabular}

\noindent \\Examples:\\
Je dois faire du jogging.
