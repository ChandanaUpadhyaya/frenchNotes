\section{Verbs and Conjugations}

There are three regular types of verbs in French:
\begin{enumerate}
\item{\textbf{Regular ER Verbs:} e.g. parler, aimer, jouer etc.}
\item{\textbf{Regular IR Verbs:} e.g. finir, \'etablir etc.}
\item{\textbf{Regular RE Verbs:} e.g. attendre, vendre}
\end{enumerate}
Other than the above, there are irregular verbs such as \^Etre, Conna\^itre, Avoir.
Their conjugation methods are unique to the verbs themselves. 

\subsection{Conjugation Pattern for Regular ER Verbs}

\subsubsection{Parler (to speak)}
\begin{tabular}{| l | r | l |}
\hline
Pronoun 	& 	Ending 	& 	Parler	\\ 	\hline
Je		&	-e	& 	parle	\\ 	\hline
Tu		&	-es	&	parles	\\	\hline
Il/Elle		&	-e	&	parle	\\	\hline
Nous		&	-ons	&	parlons	\\	\hline
Vous		&	-ez	&	parlez	\\	\hline
Ils/Elles	&	-ent	&	parlent	\\	\hline
\end{tabular}

\subsubsection{List of Regular ER Verbs}

\begin{longtable}{| l | l | l |}
\hline
Verb 		& Meaning 		& Pronunciation	\\
\hline
\endhead
parler		& to speak		& paarl			\\ 	\hline
jouer		& to play		& zooay			\\	\hline
aimer		& to like		& aimay			\\	\hline
\'ecouter	& to listen		& aikootey		\\	\hline
associer	& to associate		& associay		\\	\hline
regarder	& to look		& ghugarday		\\	\hline
tourner		& to turn		& toohnay		\\	\hline
fermer		& to close		& fahmay		\\	\hline
habiter		& to live		& aabitey		\\	\hline
adorer 		& to love		& adohay		\\	\hline
chercher	& to search		& shershay		\\	\hline
travailler	& to work		& travaiyay		\\	\hline
rester & to stay \\ \hline
\end{longtable}

\subsection{Conjugation pattern for -GER Verbs}
These are also called spelling change verbs. The conjugation pattern is
similar to regular -er verbs, but the only difference is the conjugation
with \emph{Nous}. 

\subsubsection{Manger (to eat)}
\begin{tabular}{| l | l |}
\hline
Pronoun 	& 	Conjugation	\\
Je		&	mange		\\
Tu		&	manges		\\
Il/Elle		&	mange		\\
Nous		&	mangeons	\\
Vous		&	mangez		\\
Ils/Elles	&	mangent		\\
\hline
\end{tabular}

\subsubsection{List of -GER verbs}
\begin{longtable}{| l | l | l |}
\hline
Verb 		& Meaning 		& Pronunciation	\\
\hline
\endhead
voyager 	& to travel		& voyajay		\\	\hline
manger		& to eat		& manjay		\\	\hline
arranger	& to arrange		& amanjay		\\	\hline
\end{longtable}


\subsection{Conjugation pattern for -ELER Verbs}
These are also called Stem-changing verbs. The ending pattern is similar
to regular -er verbs, but the ``ll'' in spelling is different in conjugations.

\subsubsection{Appeler (to call)}
\begin{tabular}{| l | l |}
\hline
Pronoun 	& 	Conjugation	\\
J'		&	appelle		\\
Tu		&	appelles	\\
Il/Elle		&	appelle		\\
Nous		&	appelons	\\
Vous		&	appelez		\\
Ils/Elles	&	appellent	\\
\hline
\end{tabular}

\subsubsection{List of -ELER verbs}
\begin{longtable}{| l | l | l |}
\hline
Verb 		& Meaning 		& Pronunciation	\\
\hline
\endhead
appeler 	& to call		& aapaylay	\\
\'epeler 	& to spell		& apaylay	\\
\hline
\end{longtable}

\subsection{Other Irregular Verbs}

\subsubsection{S'appeler (to call oneself)}
\begin{tabular}{| l | l |}
\hline
Pronoun 	& 	Conjugation	\\
Je		&	m'appelle	\\
Tu		&	t'appelles	\\
Il/Elle		&	s'appelle	\\
Nous		&	nous appelons	\\
Vous		&	vous appelez	\\
Ils/Elles	&	s'appellent	\\
\hline
\end{tabular}

\subsubsection{\^Etre (to be)}
\begin{tabular}{| l | l |}
\hline
Pronoun 	& 	\^Etre	\\
Je		&	suis	\\
Tu		&	es	\\
Il/Elle		&	est	\\
Nous		&	sommes	\\
Vous		&	\^etes	\\
Ils/Elles	&	sont	\\
\hline
\end{tabular}

\subsubsection{Avoir (to have)}
\begin{tabular}{| l | l |}
\hline
Pronoun 	& 	Avoir	\\
Je		&	ai	\\
Tu		&	as	\\
Il/Elle		&	a	\\
Nous		&	avons	\\
Vous		&	avez	\\
Ils/Elles	&	ont	\\
\hline
\end{tabular}

\subsubsection{Connaitre (to know)}
\begin{tabular}{| l | l |}
\hline
Pronoun 	& 	Connaitre	\\
Je		&	connais		\\
Tu		&	connais		\\
Il/Elle		&	conna\^it	\\
Nous		&	connaisons	\\
Vous		&	connaissez	\\
Ils/Elles	&	connaissent	\\
\hline
\end{tabular}

\subsubsection{Comprendre (to understand)}
\begin{tabular}{| l | l |}
\hline
Pronoun 	& 	Conjugation	\\
Je		&	comprends	\\
Tu		&	comprends	\\
Il/Elle		&	comprend	\\
Nous		&	comprenons	\\
Vous		&	comprenez	\\
Ils/Elles	&	comprennent	\\
\hline
\end{tabular}

\subsubsection{Lire (to read)}
\begin{tabular}{| l | l |}
\hline
Pronoun 	& 	Conjugation	\\
Je		&	lis		\\
Tu		&	lis		\\
Il/Elle		&	lit		\\
Nous		&	lisons		\\
Vous		&	lisez		\\
Ils/Elles	&	lisent		\\
\hline
\end{tabular}

\subsubsection{\'Ecrire (to write)}
\begin{tabular}{| l | l |}
\hline
Pronoun 	& 	Conjugation	\\
Je		&	\'ecris		\\
Tu		&	\'ecris		\\
Il/Elle		&	\'ecrit		\\
Nous		&	\'ecrivons	\\
Vous		&	\'ecrivez	\\
Ils/Elles	&	\'ecrivent	\\
\hline
\end{tabular}


\subsubsection{Voir (to see)}
\begin{tabular}{| l | l |}
\hline
Pronoun 	& 	Conjugation	\\
Je		&	voi\ul{s}	\\
Tu		&	voi\ul{s}	\\
Il/Elle		&	voi\ul{t}	\\
Nous		&	voyons		\\
Vous		&	voyez		\\
Ils/Elles	&	voi\ul{ent}	\\
\hline
\end{tabular}

\subsubsection{Aller (to go)}
\begin{tabular}{| l | l |}
\hline
Pronoun 	& 	Conjugation	\\
Je		&	vais		\\
Tu		&	vas		\\
Il/Elle		&	va		\\
Nous		&	allons		\\
Vous		&	allez		\\
Ils/Elles	&	vont		\\
\hline
\end{tabular}

\subsubsection{Venir (to come)}
\begin{tabular}{| l | l |}
\hline
Pronoun 	& 	Conjugation	\\
Je		&	viens		\\
Tu		&	viens		\\
Il/Elle		&	vient		\\
Nous		&	venons		\\
Vous		&	venez		\\
Ils/Elles	&	viennent		\\
\hline
\end{tabular}

\subsubsection{Faire (to do / to make)}
\begin{tabular}{| l | l |}
\hline
Pronoun 	& 	Conjugation	\\
Je		&	fais		\\
Tu		&	fais		\\
Il/Elle		&	fait		\\
Nous		&	faisons		\\
Vous		&	faites		\\
Ils/Elles	&	font		\\
\hline
\end{tabular}
